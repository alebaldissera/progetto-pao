\section{Descrizione delle chiamate polimorfe}

La gerarchia, come già visto nella descrizione dei tipi, fa uso dei metodi virtuali e quindi delle chiamate polimorfe. L'uso di queste chiamate è stato 
sfruttato principalmente per fornire delle informazioni specifiche per ogni tipo della gerarchia all'utente o per differenziare il comportamento di un metodo a seconda del tipo usato. 
I metodi virtuali puri \emph{getSize()} e \emph{getAllSize()} sono stati specializzati per ogni sottotipo della gerarchia. Nel tipo \emph{Katalog::Directory} 
restituiscono entrambi lo stesso risultato, mentre per i sottotipi di quest'ultimo restituiscono rispettivamente la dimensione del file e la dimensione del file sommata alla dimensione
di tutti gli elementi contenuti in esso (in modo ricorsivo). Il metodo \emph{getInfo()} permette di restituire all'utente delle informazioni specifiche per ogni tipo della gerarchia;
in effetti queste informazioni vengono poi usate per arricchire l'interfaccia utente.