\section{Descrizione dei tipi}

Tutte le seguenti classi sono state raccolte all'interno di uno spazio
dei nomi chiamato \emph{Katalog}, questo per tenere separate tutte
le definizioni dei tipi da eventuali conflitti di nomi. L'accesso 
a queste classi avviene tramite l'apertura dello spazio dei nomi attraverso
l'istruzione:
\begin{lstlisting}[language=C++]
    using namespace Katalog;
\end{lstlisting}
Oppure attraverso l'apertura di una parte del namespace con l'istruzione:
\begin{lstlisting}[language=C++]
    using Katalog::DeepPtr;
\end{lstlisting}
Oppure utilizzando l'operatore di scoping ogni volta che si vuole accedere
ad una definizione interna al namespace.

\subsection{Class \emph{Vector}}

La classe \emph{Vector} rappresenta un array a dimensione variabile che
gestisce automaticamente la memoria, riallocando quando necessario
l'array di dati e copiando di conseguenza tutti i valori dall'array
più piccolo a quello più grande. In particolare \emph{Vector} è definito
come un template di classe che permette di riutilizzare lo stesso codice
per qualsiasi tipo di dati. La definizione e l'implementazione di questa 
classe si trova in un unico file, essendo un template, chiamato Vector.h.

\subsection{Class \emph{DeepPtr}}

La classe \emph{DeepPtr} rappresenta un puntatore ad oggetti polimorfi
con gestione automatica della memoria tramite il contratto standard
\emph{clone()} delle classi polimorfe. Anche questo tipo è definito come
un template, permettendo il suo utilizzo con qualsiasi gerarchia polimorfa
che supporta il metodo \emph{clone()}. Come per la classe precedente, anche 
questa è definita ed implementata in un unico file chiamato DeepPtr.h

\subsection{Classe \emph{BaseNode}}

La classe \emph{BaseNode} è una classe virtuale astratta: ciò significa 
che mancano le implementazioni di alcuni metodi virtuali e per questo 
non è istanziabile. Questa rappresenta la base di tutta la gerarchia di
tipi su cui si basa il progetto. \emph{BaseNode} implementa alcune
caratteristiche comuni a tutte le classi derivate come la gestione del nome
virtuale, la gestione della path sul file system e la gestione dei sotto-file.
Il metodo virtuale puro \emph{clone()} si occupa di copiare l'oggetto
ogni volta che è necessario. La gestione della memoria è lasciata all'oggetto
\emph{DeepPtr} visto precedentemente. Gli altri metodi virtuali puri saranno 
discussi separatamente per ogni classe. Questa classe è definita nel file BaseNode.h ed è implementata nel file BaseNode.cpp.
L'attributo directoryOpened indica se sono visibili i sotto-file ed è solamente usata dalla GUI.

\subsection{Classe \emph{Directory}}

La classe \emph{Directory}, derivata da \emph{BaseNode}, definisce il comportamento di una cartella virtuale,
definita dal nome e dai file in essa contenuti. I metodi \emph{getSize()} e \emph{getAllSize()}
si comportano allo stesso modo e ritornano la dimensione di tutti gli elementi contenuti nella cartella virtuale.
L'attributo path definito in \emph{BaseNode} è sempre nullo e non viene preso in considerzione. 
Il metodo \emph{getInfo()} ritorna una stringa formattata contenente il numero di file contenuti e la dimensione totale.

\subsection{Classi \emph{Photo}, \emph{Audio} e \emph{Video}}

Queste classi, derivate da \emph{Directory}, si occupano di rappresentare rispettivamente immagini, file audio e video.
Le classi contengono alcuni attributi specifici per ogni file, usati per dare informazioni all'utente sul file attraverso il metodo \emph{getInfo()}.
In queste classi l'attributo path indica il percorso fisico nel file system per raggiungere il file. 
Il metodo \emph{getSize()} ritorna la dimensione del file che rappresenta, mentre il metodo \emph{getAllSize()}
ritorna la dimensione del file sommata alla dimensione di tutti i file contenuti all'interno di esso.

\subsection{Classe \emph{Catalogo}}

La classe \emph{Catalogo} rappresenta l'interfaccia del modello dell'applicazione.
Essa è composta dalla radice della struttura dei file (istanziata come una \emph{Directory} senza nome), 
e da un attributo di tipo bool che indica se la struttura ha subito delle modifiche.
L'interfaccia pubblica di questa classe offre tutto il necessario per manipolare la struttura dei file
contenuta nel catalogo, nascondendo completamente il comportamento della gerarchia dall'esterno.
A livello di \emph{Catalogo} tutti i file sono identificati da una path virtuale rappresentata su una stringa
e costruita concatenando i nomi dei file a cui bisogna accedere per arrivare a quello interessato, separandoli da uno /.
Per elaborare la path virtuale si usano le regex o espressioni regolari che si occupano di fare tutti i controlli di validità dei nomi dei file e separare i vari elementi compresi tra due /.
