\section{Bug e problemi conosciuti}

Il software non è esente da bug o problemi di vario tipo. In primis la finestra di caricamento (classe LoadingWindow) non sempre viene renderizzata all'avvio del 
software. In particolare succede quando il catalogo multimediale inizia a diventare piuttosto grande e il caricamento delle informazioni di tutti i file inizia a diventare oneroso.
Nella fase di chiusura invece, anche con parecchi file inseriti, non sono stati riscontrati rallentamenti significativi. \newline
Il secondo problema è stato riscontrato nella fase di importazione di molti file, specialmente se in formati con scarsa compressione. In questo caso il programma si blocca per 
qualche secondo per poi riprendere il normale funzionamento. Questi due problemi riscontrati sono legati alla lettura dei file che rallenta l'importazione di nuovi file o dell'intero
catalogo. \newline
Terzo problema è legato al caricamento delle anteprime della griglia. In questo caso, venendo lanciati un thread separato per anteprima (solo nel caso di immagini),
vengono caricate in memoria molte immagini che rallenta l'accesso alla memoria di massa richiedendo molti file separati. Questo comporta un'uso eccessivo di memoria nella fase di
caricamento delle anteprime che potrebbe comportare instabilità del sistema. Questo problema è stato riscontrato testando il software con circa 200 fotografie da circa 10MB l'una.
La memoria viene comunque liberata correttamente e il thread distrutto appena l'anteprima è caricata. \newline
Infine cito un bug legato al formato di file utilizzato per salvare il catalogo. Essendo UTF-8 la codifica utilizzata dai file XML, diversa dalla codifica
del testo del C++, quando si trova di fronte stringhe in cui ci sono dei caratteri speciali (dai nostri test accade con le vocali accentate), l'esportazione e l'importazione 
del catalogo dal file non permette la lettura corretta di quei caratteri, nonostante il software riesca a gestirli correttamente. Il software comunque non termina, convertendo il 
file in una cartella (a livello di catalogo) ma nonostante questo tutti i suoi sotto file risulteranno irraggiungibili.