\section{Descrizione del formato dei file}

Per salvare la struttura del catalogo sul file è stato scelto il formato XML.
Questo perchè XML usa una struttura gerarchica come quella di file e cartelle. 
I dati sono rappresentati da "TAG" che sono stati usati per differenziare 
il tipo di dato da rappresentare e gli "attributi" per descrivere proprietà dei dati.
In particolare il TAG \emph{Katalog} è usato per aprire il documento, in quanto tutti i file XML necessitano di un tag di apertura (e rispettivamente di chiusura).
All'interno di questo TAG sono rappresentati tutti i dati del catalogo. 
Nella struttura del file è previsto un TAG chiamato "KatalogVersion" che è stato inserito appositamente 
per il versionamento del salvataggio, permettendo di distinguere facilmente il formato specifico 
della struttura XML nel caso nelle future versioni del software questa debbba essere cambiata. 
Il TAG "FileStructure" contiene le informazioni per ricreare l'intero catalogo. Al suo interno a sua volta si possono 
trovare i tag "KFile" e "KDirectory". I due tag hanno degli attributi:
\begin{itemize}
    \item "Name" che indica il nome del file/cartella sul catalogo;
    \item "PathToDisk" che indica la path del file specifico su disco ed è disponibile solo per i TAG "KFile" altrimenti verrà ignorato;
    \item "FileType" che indica il tipo di file ed è disponibile solo per i TAG "KFile" altrimenti verrà ingnorato;
    \item "IsOpen" che indica se nell'ultima sessione di utilizzo del catalogo è stata visualizzata la lista dei sottofile.
\end{itemize}
Infine all'interno di "KFile" e "KDirectory" si può trovare un TAG chiamato "Childs" che a sua volta conterrà di "KFile" o "KDirectory" ed è usato per 
indicare i sotto-file. Si è scelto di non inserire i sotto-file direttamnte all'interno del tag
del file o directory per semplificare un'eventuale aggiunta di altre informazioni relative all'oggetto rappresentato. 
La gestione della lettura e scrittura del file XML rappresentante i dati del catalogo, avviene tramite la classe \emph{Katalog::IOManager} che non necessita di essere istanziata per 
poter essere usata. Questa infatti espone due metodi pubblici statici che si occupano di leggere e scrivere il file; rispettivamente chiamati \emph{importCatalogFromFile} e \emph{exportCatalogToFile}.
Il metodo di lettura del file accetta come parametro una stringa (\emph{std::string}) che indica la path su disco del file da leggere. A partire dalla path, aprirà il file e 
ne interpreterà il contenuto ritornando l'intero catalogo già costruito e popolato. Il metodo di scrittura del file invece accetta come parametri il catalogo (\emph{Katalog::Catalogo}) e 
la una string (\emph{std::string}) che indica la path su disco, dove salvare il file. Questo metodo si occuperà di leggere tutta la struttura del catalogo e di scrivere il file nel formato corretto.
Per semplicità è stato sfruttato il supporto nativo del framework Qt per leggere e scrivere i file XML.