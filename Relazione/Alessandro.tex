\documentclass[11pt, letterpaper]{article}
\usepackage[utf8]{inputenc}
\usepackage[margin=1in]{geometry}
\usepackage{listings}
\renewcommand{\contentsname}{Indice}

\title{\textbf{Progetto Programmazione ad oggetti \\ Sviluppo del software Katalog}}
\author{Alessandro Baldissera Mat:1216742 \\ Sviluppo in coppia con Nicolas Alberti Mat:1224454}
\date{Anno Accademico 2020/2021}

\begin{document}
\maketitle

\begin{abstract}
    Si vuole creare un software che permetta di catalogare e riprodurre
    dei file multimediali, in particolare foto, video e file audio.
    L'organizzazione dei file deve avvenire tramite una struttura gerarchica 
    simile a quella di un file-system.
    La struttura si compone di quattro componenti: cartelle, foto, audio e video.
    In particolare foto, audio e video sono a loro volta delle cartelle e 
    quindi possono contenere dei file.
    L'utente potrà navigare nella struttura attraverso una visualizzazione a griglia
    e attraverso una visualizzazione dell'albero. Inoltre dovrà poter riprodurre
    i file multimediali che inserisce nel catalogo. Il salvataggio del catalogo
    deve essere del tutto trasparente all'utente e avvenire in automatico alla
    chiusura del software ma l'utente dovrà avere la possibilità di salvare manualmente
    in qualsiasi momento.
\end{abstract}

\tableofcontents{}
\newpage

\section{Descrizione dei tipi}

Tutte le seguenti classi sono state raccolte all'interno di uno spazio
dei nomi chiamato \emph{Katalog}, questo per tenere separate tutte
le definizioni dei tipi da eventuali conflitti di nomi. L'accesso 
a queste classi avviene tramite l'apertura dello spazio dei nomi attraverso
l'istruzione:
\begin{lstlisting}[language=C++]
    using namespace Katalog;
\end{lstlisting}
Oppure attraverso l'apertura di una parte del namespace con l'istruzione:
\begin{lstlisting}[language=C++]
    using Katalog::DeepPtr;
\end{lstlisting}
Oppure utilizzando l'operatore di scoping ogni volta che si vuole accedere
ad una definizione interna al namespace.

\subsection{Class \emph{Vector}}

La classe \emph{Vector} rappresenta un array a dimensione variabile che
gestisce automaticamente la memoria, riallocando quando necessario
l'array di dati e copiando di conseguenza tutti i valori dall'array
più piccolo a quello più grande. In particolare \emph{Vector} è definito
come un template di classe che permette di riutilizzare lo stesso codice
per qualsiasi tipo di dati. La definizione e l'implementazione di questa 
classe si trova in un unico file, essendo un template, chiamato Vector.h.

\subsection{Class \emph{DeepPtr}}

La classe \emph{DeepPtr} rappresenta un puntatore ad oggetti polimorfi
con gestione automatica della memoria tramite il contratto standard
\emph{clone()} delle classi polimorfe. Anche questo tipo è definito come
un template, permettendo il suo utilizzo con qualsiasi gerarchia polimorfa
che supporta il metodo \emph{clone()}. Come per la classe precedente, anche 
questa è definita ed implementata in un unico file chiamato DeepPtr.h

\subsection{Classe \emph{BaseNode}}

La classe \emph{BaseNode} è una classe virtuale astratta: ciò significa 
che mancano le implementazioni di alcuni metodi virtuali e per questo 
non è istanziabile. Questa rappresenta la base di tutta la gerarchia di
tipi su cui si basa il progetto. \emph{BaseNode} implementa alcune
caratteristiche comuni a tutte le classi derivate come la gestione del nome
virtuale, la gestione della path sul file system e la gestione dei sotto-file.
Il metodo virtuale puro \emph{clone()} si occupa di copiare l'oggetto
ogni volta che è necessario. La gestione della memoria è lasciata all'oggetto
\emph{DeepPtr} visto precedentemente. Gli altri metodi virtuali puri saranno 
discussi separatamente per ogni classe. Questa classe è definita nel file BaseNode.h ed è implementata nel file BaseNode.cpp.
L'attributo directoryOpened indica se sono visibili i sotto-file ed è solamente usata dalla GUI.

\subsection{Classe \emph{Directory}}

La classe \emph{Directory}, derivata da \emph{BaseNode}, definisce il comportamento di una cartella virtuale,
definita dal nome e dai file in essa contenuti. I metodi \emph{getSize()} e \emph{getAllSize()}
si comportano allo stesso modo e ritornano la dimensione di tutti gli elementi contenuti nella cartella virtuale.
L'attributo path definito in \emph{BaseNode} è sempre nullo e non viene preso in considerzione. 
Il metodo \emph{getInfo()} ritorna una stringa formattata contenente il numero di file contenuti e la dimensione totale.

\subsection{Classi \emph{Photo}, \emph{Audio} e \emph{Video}}

Queste classi, derivate da \emph{Directory}, si occupano di rappresentare rispettivamente immagini, file audio e video.
Le classi contengono alcuni attributi specifici per ogni file, usati per dare informazioni all'utente sul file attraverso il metodo \emph{getInfo()}.
In queste classi l'attributo path indica il percorso fisico nel file system per raggiungere il file. 
Il metodo \emph{getSize()} ritorna la dimensione del file che rappresenta, mentre il metodo \emph{getAllSize()}
ritorna la dimensione del file sommata alla dimensione di tutti i file contenuti all'interno di esso.

\subsection{Classe \emph{Catalogo}}

La classe \emph{Catalogo} rappresenta l'interfaccia del modello dell'applicazione.
Essa è composta dalla radice della struttura dei file (istanziata come una \emph{Directory} senza nome), 
e da un attributo di tipo bool che indica se la struttura ha subito delle modifiche.
L'interfaccia pubblica di questa classe offre tutto il necessario per manipolare la struttura dei file
contenuta nel catalogo, nascondendo completamente il comportamento della gerarchia dall'esterno.
A livello di \emph{Catalogo} tutti i file sono identificati da una path virtuale rappresentata su una stringa
e costruita concatenando i nomi dei file a cui bisogna accedere per arrivare a quello interessato, separandoli da uno /.
Per elaborare la path virtuale si usano le regex o espressioni regolari che si occupano di fare tutti i controlli di validità dei nomi dei file e separare i vari elementi compresi tra due /.


\section{Descrizione delle chiamate polimorfe}

La gerarchia, come già visto nella descrizione dei tipi, fa uso dei metodi virtuali e quindi delle chiamate polimorfe. L'uso di queste chiamate è stato 
sfruttato principalmente per fornire delle informazioni specifiche per ogni tipo della gerarchia all'utente o per differenziare il comportamento di un metodo a seconda del tipo usato. 
I metodi virtuali puri \emph{getSize()} e \emph{getAllSize()} sono stati specializzati per ogni sottotipo della gerarchia. Nel tipo \emph{Katalog::Directory} 
restituiscono entrambi lo stesso risultato, mentre per i sottotipi di quest'ultimo restituiscono rispettivamente la dimensione del file e la dimensione del file sommata alla dimensione
di tutti gli elementi contenuti in esso (in modo ricorsivo). Il metodo \emph{getInfo()} permette di restituire all'utente delle informazioni specifiche per ogni tipo della gerarchia;
in effetti queste informazioni vengono poi usate per arricchire l'interfaccia utente.

\section{Descrizione del formato dei file}

Per salvare la struttura del catalogo sul file è stato scelto il formato XML.
Questo perchè XML usa una struttura gerarchica come quella di file e cartelle. 
I dati sono rappresentati da "TAG" che sono stati usati per differenziare 
il tipo di dato da rappresentare e gli "attributi" per descrivere proprietà dei dati.
In particolare il TAG \emph{Katalog} è usato per aprire il documento, in quanto tutti i file XML necessitano di un tag di apertura (e rispettivamente di chiusura).
All'interno di questo TAG sono rappresentati tutti i dati del catalogo. 
Nella struttura del file è previsto un TAG chiamato "KatalogVersion" che è stato inserito appositamente 
per il versionamento del salvataggio, permettendo di distinguere facilmente il formato specifico 
della struttura XML nel caso nelle future versioni del software questa debbba essere cambiata. 
Il TAG "FileStructure" contiene le informazioni per ricreare l'intero catalogo. Al suo interno a sua volta si possono 
trovare i tag "KFile" e "KDirectory". I due tag hanno degli attributi:
\begin{itemize}
    \item "Name" che indica il nome del file/cartella sul catalogo;
    \item "PathToDisk" che indica la path del file specifico su disco ed è disponibile solo per i TAG "KFile" altrimenti verrà ignorato;
    \item "FileType" che indica il tipo di file ed è disponibile solo per i TAG "KFile" altrimenti verrà ingnorato;
    \item "IsOpen" che indica se nell'ultima sessione di utilizzo del catalogo è stata visualizzata la lista dei sottofile.
\end{itemize}
Infine all'interno di "KFile" e "KDirectory" si può trovare un TAG chiamato "Childs" che a sua volta conterrà di "KFile" o "KDirectory" ed è usato per 
indicare i sotto-file. Si è scelto di non inserire i sotto-file direttamnte all'interno del tag
del file o directory per semplificare un'eventuale aggiunta di altre informazioni relative all'oggetto rappresentato. 
La gestione della lettura e scrittura del file XML rappresentante i dati del catalogo, avviene tramite la classe \emph{Katalog::IOManager} che non necessita di essere istanziata per 
poter essere usata. Questa infatti espone due metodi pubblici statici che si occupano di leggere e scrivere il file; rispettivamente chiamati \emph{importCatalogFromFile} e \emph{exportCatalogToFile}.
Il metodo di lettura del file accetta come parametro una stringa (\emph{std::string}) che indica la path su disco del file da leggere. A partire dalla path, aprirà il file e 
ne interpreterà il contenuto ritornando l'intero catalogo già costruito e popolato. Il metodo di scrittura del file invece accetta come parametri il catalogo (\emph{Katalog::Catalogo}) e 
la una string (\emph{std::string}) che indica la path su disco, dove salvare il file. Questo metodo si occuperà di leggere tutta la struttura del catalogo e di scrivere il file nel formato corretto.
Per semplicità è stato sfruttato il supporto nativo del framework Qt per leggere e scrivere i file XML.

\section{Descrizione della GUI}

La creazione della GUI è iniziata dalla finestra principale che l'utente vedrà all'avvio del software. Questa è composta da due aree dello schermo divise verticalmente.
Partendo dal lato destro si trova un widget che mostra la struttura ad albero del catalogo. Questo widget (come viene chiamato dal framework Qt) è stato personalizzato 
appositamente per poter deselezionare completamente gli elementi premendo in un'area bianca del widget. Per ottenere questo risultato è stata derivata la classe di Qt chiamata
\emph{QTreeView} ed è stato effettuato un override del metodo che gestisce l'evento di click nel widget stesso. La soluzione è stata ispirata da una discussione trovata in rete e,
successivamente, personalizzata seguendo le nostre necessità. Inoltre la manipolazine dei file (copia-taglia-incolla) è possibile direttamente da questo widget.
Al di sotto del \emph{QTreeWidget} si trova un \emph{QPushButton} che permette di importare i file senza passare dal menu File.
Il lato sinistro dell'interfaccia utente è più complesso. Nella parte superiore è situata una QLineEdit che 
si occupa di mostrare all'utente la path in cui si trova attualmente, ma ne permette anche l'inserimento manuale per muoversi all'interno del catalogo stesso. 
A fianco si trova un pulsante che permette di passare direttamente dalla griglia delle anteprime al player.
Il resto del lato sinistro è composto da una visualizzazione a griglia definita dal widget \emph{GridView} e dal widget \emph{VideoPlayer} che cambiano dinamicamente a seconda 
del contesto in cui si trova il catalogo; in ogni caso è possibile passare da una visualizzazione ad un'altra usando il menù o delle scorciatoie da tastiera. Il menù principale
nella parte superiore della finestra contiene tutti i collegamenti alle possibili azioni con relative scorciatoie da tastiera.

\subsection{Visualizzazione a griglia}
La visualizzazione a griglia (definita in \emph{GridView.h} e implementata in \emph{GridView.cpp}) è composta nel modo seguente:
\begin{enumerate}
    \item Un widget QScrollArea che permette di scorrere il contenuto quando questo si estende anche oltre le dimensioni della finestra;
    \item Un widget fittizio il cui unico scopo è contenere un layout; inserito nella QScrollArea per poter attivare lo scorrimento;
    \item Un layout FluidLayout cioè una griglia che posiziona dinamicamente gli elementi al suo interno a seconda della dimensione degli elementi e della finestra (è un widget fornito tra gli esempi del framework Qt e quindi considerato come parte di esso);
    \item I widget personalizzati PreviewWindow che rappresentano le icone degli elementi del catalogo.
\end{enumerate}
I widget personalizzati PreviewWindow (definiti in \emph{PreviewWindow.h} e implementati in \emph{PreviewWindow.cpp}) rappresentano le icone dei singoli file e sono composti
essenzialmete da due QLabel che mostrano l'icona e il nome del file. Nel caso in cui il file che rappresentano sia una foto, mostreranno anche un'anteprima della foto stessa.
Per farlo si è scelto di usare i QThread per migliorare l'esperienza utente. Infatti il caricamento delle anteprime può diventare molto lento e quindi bloccare il programma.
Per implementare il caricamento delle anteprime attraverso i QThread si è creato un sottotipo di QThread che implementa il caricamento e il ridimensionamento dell'immagine. 
Quando il widget PreviewWindow si trova di fronte un'immagine lancerà il thread; nel caso in cui il widget venga rimosso dalla griglia prima che il thread completi il suo lavoro, 
questo verrà fermato forzatamente e successivamente deallocato. Nella griglia per 
motivi di tempo non è stata implementata la manipolazione dei file, che rimane comunque possibile attraverso la visualizzazione ad albero.

\subsection{Riproduttore multimediale}
Il widget VideoPlayer (definito in \emph{VideoPlayer.h} e implementato in \emph{VideoPlayer.cpp}) si occupa di riprodurre tutti i file multimediali ed è composto principalmente 
da due parti:
\begin{enumerate}
    \item Parte superiore che si occupa di dare un feedback grafico sul file in riproduzione ed è composta da:
    \begin{itemize}
        \item Una QLabel usata par immagini e file audio (in questo caso mostra un'icona);
        \item Un QVideoWidget che permette di riprodurre un video;
    \end{itemize}
    \item Parte inferiore composta dai controlli del player.
\end{enumerate}
I widget della parte superiore sono disposti su un QStakedLayout che permette di sovrappore i widget in "strati" e scegliere quale widget si vuole visualizzare.
Mentre le immagini vengono visualizzate direttamnte sulla QLabel descritta precedentemente, selezionandone il giusto layer nel layout appena descritto, per video e audio
la gestione è delegata ad un QMediaPlayer (potrebbe riprodurre anche le immagini ma per un bug causato dai codec si è scelto per una gestione manuale).
Il QMediaPlayer infine si appoggia al QVideoWidget per mostrare a schermo i video. Si potrebbe far uso di una QPlayList per raccogliere tutti i file multimediali
e gestirli direttamente dal QMediaPlayer ma come già introdotto i codec su linux quando il file media termina la sua riproduzione renderizzano uno schermo nero, non permettendo 
di vedere le immagini.

\section{Compilazione del codice e testing}

Il software è stato sviluppato e testato utilizzando i seguenti sistemi:
\begin{itemize}
    \item Windows 10 Pro, MinGW 7.3, Qt 5.12.10, compilato in 64 bit
    \item Windows 10 Pro, MSVC2015, Qt 5.12.10, compilato in 64 bit
    \item ArchLinux 5.10, GCC 10.2, Qt 5.15.2, compilato in 64 bit
    \item ArchLinux 5.10, CLANG 11, Qt 5.15.2, compilato in 64 bit
\end{itemize}
Inoltre sono stati svolti dei test attraverso il software \emph{Valgrind 3.16}, riscontrando memory leak riconducibili al framework Qt.
Come IDE è stato utilizzato QtCreator 4.13.
\newline 
Per compilare su sistemi linux è necessario spostarsi nella cartella contenente i sorgenti e digitare i seguenti comandi:
\begin{lstlisting}[language=bash]
    qmake Katalog.pro
    make
\end{lstlisting}
oppure per sistemi multicore, per velocizzare la fase di compilazione anche:
\begin{lstlisting}[language=bash]
    qmake Katalog.pro
    make -j4
\end{lstlisting}
dove si può sostiturie il numero \emph{4} con il numero di core del sistema; con il parametro \emph{-j}, make si occuperà automaticamente di 
distribuire la compilazione su tutti i core disponibili. Infine per eseguire il programma basterà digitare:
\begin{lstlisting}[language=bash]
    ./Katalog
\end{lstlisting}

\section{Tempistiche di progetto}

Il progetto ha richiesto in totale 53 ore, superando le 50 ore indicate. Principalmente il tempo extra è stato impiegato per migliorare i tempi di caricamento delle anteprime
dei file e per risolvere alcuni problemi emersi nel player già citati precedentemente. In particolare il tempo a disposizione è stato impiegato in questo modo:
\begin{itemize}
    \item 5 ore per analizzare il problema e organizzare il lavoro
    \item 4 ore per la progettazione del modello
    \item 6 ore per la realizzazione del modello
    \item 4 ore per la progettazione della GUI
    \item 6 ore per apprendimento del framework
    \item 12 ore per la realizzazione della GUI
    \item 10 ore per test e debug
    \item 6 ore per l'ottimizzazione dell'esecuzione
\end{itemize}
La fase di ottimizzazione ha richiesto diverso tempo, in quanto integrando l'uso dei thread, sono state fatte modifiche pittosto profonde alle parti coinvolte della GUI.


\section{Divisione dei compiti}

Il progetto è stato svolto in coppia. I compiti sono stati divisi in modo tale da svolgere ognuno circa metà lavoro sul modello e metà sulla GUI.
In particolare il lavoro è stato diviso nel modo seguente. \newline
Lavoro svolto da Alessandro:
\begin{itemize}
    \item Sviluppo del contenitore (classe Vector)
    \item Sviluppo del puntatore smart (classe DeepPtr)
    \item Sviluppo delle classi istanziabili della gerarchia (classi Directory, Photo, Audio, Video)
    \item Sviluppo del gestore del file (classe IOManager)
    \item Sviluppo dei widget (classi MainWindow e LoadingWindow, DeselectableTreeWidget)
\end{itemize}
Lavoro svolto da Nicolas:
\begin{itemize}
    \item Svilippo della classe astratta della gerarchia (classe BaseNode)
    \item Sviluppo dell'interfaccia del modello (classe Catalogo)
    \item Svliluppo dei widget (classi FlowLayout, GridView, PreviewWindow, VideoPlayer)
\end{itemize}
Lo sviluppo del controller invece è stato condiviso da entrambi, aggiungendo a mano a mano il codice e le funzionalità che si implementavano separatamente.
Il codice prodotto è stato comunque revisionato reciprocamente. 

\section{Bug e problemi conosciuti}

Il software non è esente da bug o problemi di vario tipo. In primis la finestra di caricamento (classe LoadingWindow) non sempre viene renderizzata all'avvio del 
software. In particolare succede quando il catalogo multimediale inizia a diventare piuttosto grande e il caricamento delle informazioni di tutti i file inizia a diventare oneroso.
Nella fase di chiusura invece, anche con parecchi file inseriti, non sono stati riscontrati rallentamenti significativi. \newline
Il secondo problema è stato riscontrato nella fase di importazione di molti file, specialmente se in formati con scarsa compressione. In questo caso il programma si blocca per 
qualche secondo per poi riprendere il normale funzionamento. Questi due problemi riscontrati sono legati alla lettura dei file che rallenta l'importazione di nuovi file o dell'intero
catalogo. \newline
Terzo problema è legato al caricamento delle anteprime della griglia. In questo caso, venendo lanciati un thread separato per anteprima (solo nel caso di immagini),
vengono caricate in memoria molte immagini che rallenta l'accesso alla memoria di massa richiedendo molti file separati. Questo comporta un'uso eccessivo di memoria nella fase di
caricamento delle anteprime che potrebbe comportare instabilità del sistema. Questo problema è stato riscontrato testando il software con circa 200 fotografie da circa 10MB l'una.
La memoria viene comunque liberata correttamente e il thread distrutto appena l'anteprima è caricata. \newline
Infine cito un bug legato al formato di file utilizzato per salvare il catalogo. Essendo UTF-8 la codifica utilizzata dai file XML, diversa dalla codifica
del testo del C++, quando si trova di fronte stringhe in cui ci sono dei caratteri speciali (dai nostri test accade con le vocali accentate), l'esportazione e l'importazione 
del catalogo dal file non permette la lettura corretta di quei caratteri, nonostante il software riesca a gestirli correttamente. Il software comunque non termina, convertendo il 
file in una cartella (a livello di catalogo) ma nonostante questo tutti i suoi sotto file risulteranno irraggiungibili.

\end{document}

