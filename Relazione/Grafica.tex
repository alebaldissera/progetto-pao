\section{Descrizione della GUI}

La creazione della GUI è iniziata dalla finestra principale che l'utente vedrà all'avvio del software. Questa è composta da due aree dello schermo divise verticalmente.
Partendo dal lato destro si trova un widget che mostra la struttura ad albero del catalogo. Questo widget (come viene chiamato dal framework Qt) è stato personalizzato 
appositamente per poter deselezionare completamente gli elementi premendo in un'area bianca del widget. Per ottenere questo risultato è stata derivata la classe di Qt chiamata
\emph{QTreeView} ed è stato effettuato un override del metodo che gestisce l'evento di click nel widget stesso. La soluzione è stata ispirata da una discussione trovata in rete e,
successivamente, personalizzata seguendo le nostre necessità. Inoltre la manipolazine dei file (copia-taglia-incolla) è possibile direttamente da questo widget.
Al di sotto del \emph{QTreeWidget} si trova un \emph{QPushButton} che permette di importare i file senza passare dal menu File.
Il lato sinistro dell'interfaccia utente è più complesso. Nella parte superiore è situata una QLineEdit che 
si occupa di mostrare all'utente la path in cui si trova attualmente, ma ne permette anche l'inserimento manuale per muoversi all'interno del catalogo stesso. 
A fianco si trova un pulsante che permette di passare direttamente dalla griglia delle anteprime al player.
Il resto del lato sinistro è composto da una visualizzazione a griglia definita dal widget \emph{GridView} e dal widget \emph{VideoPlayer} che cambiano dinamicamente a seconda 
del contesto in cui si trova il catalogo; in ogni caso è possibile passare da una visualizzazione ad un'altra usando il menù o delle scorciatoie da tastiera. Il menù principale
nella parte superiore della finestra contiene tutti i collegamenti alle possibili azioni con relative scorciatoie da tastiera.

\subsection{Visualizzazione a griglia}
La visualizzazione a griglia (definita in \emph{GridView.h} e implementata in \emph{GridView.cpp}) è composta nel modo seguente:
\begin{enumerate}
    \item Un widget QScrollArea che permette di scorrere il contenuto quando questo si estende anche oltre le dimensioni della finestra;
    \item Un widget fittizio il cui unico scopo è contenere un layout; inserito nella QScrollArea per poter attivare lo scorrimento;
    \item Un layout FluidLayout cioè una griglia che posiziona dinamicamente gli elementi al suo interno a seconda della dimensione degli elementi e della finestra (è un widget fornito tra gli esempi del framework Qt e quindi considerato come parte di esso);
    \item I widget personalizzati PreviewWindow che rappresentano le icone degli elementi del catalogo.
\end{enumerate}
I widget personalizzati PreviewWindow (definiti in \emph{PreviewWindow.h} e implementati in \emph{PreviewWindow.cpp}) rappresentano le icone dei singoli file e sono composti
essenzialmete da due QLabel che mostrano l'icona e il nome del file. Nel caso in cui il file che rappresentano sia una foto, mostreranno anche un'anteprima della foto stessa.
Per farlo si è scelto di usare i QThread per migliorare l'esperienza utente. Infatti il caricamento delle anteprime può diventare molto lento e quindi bloccare il programma.
Per implementare il caricamento delle anteprime attraverso i QThread si è creato un sottotipo di QThread che implementa il caricamento e il ridimensionamento dell'immagine. 
Quando il widget PreviewWindow si trova di fronte un'immagine lancerà il thread; nel caso in cui il widget venga rimosso dalla griglia prima che il thread completi il suo lavoro, 
questo verrà fermato forzatamente e successivamente deallocato. Nella griglia per 
motivi di tempo non è stata implementata la manipolazione dei file, che rimane comunque possibile attraverso la visualizzazione ad albero.

\subsection{Riproduttore multimediale}
Il widget VideoPlayer (definito in \emph{VideoPlayer.h} e implementato in \emph{VideoPlayer.cpp}) si occupa di riprodurre tutti i file multimediali ed è composto principalmente 
da due parti:
\begin{enumerate}
    \item Parte superiore che si occupa di dare un feedback grafico sul file in riproduzione ed è composta da:
    \begin{itemize}
        \item Una QLabel usata par immagini e file audio (in questo caso mostra un'icona);
        \item Un QVideoWidget che permette di riprodurre un video;
    \end{itemize}
    \item Parte inferiore composta dai controlli del player.
\end{enumerate}
I widget della parte superiore sono disposti su un QStakedLayout che permette di sovrappore i widget in "strati" e scegliere quale widget si vuole visualizzare.
Mentre le immagini vengono visualizzate direttamnte sulla QLabel descritta precedentemente, selezionandone il giusto layer nel layout appena descritto, per video e audio
la gestione è delegata ad un QMediaPlayer (potrebbe riprodurre anche le immagini ma per un bug causato dai codec si è scelto per una gestione manuale).
Il QMediaPlayer infine si appoggia al QVideoWidget per mostrare a schermo i video. Si potrebbe far uso di una QPlayList per raccogliere tutti i file multimediali
e gestirli direttamente dal QMediaPlayer ma come già introdotto i codec su linux quando il file media termina la sua riproduzione renderizzano uno schermo nero, non permettendo 
di vedere le immagini.