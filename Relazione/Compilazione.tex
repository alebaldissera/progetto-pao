\section{Compilazione del codice e testing}

Il software è stato sviluppato e testato utilizzando i seguenti sistemi:
\begin{itemize}
    \item Windows 10 Pro, MinGW 7.3, Qt 5.12.10, compilato in 64 bit
    \item Windows 10 Pro, MSVC2015, Qt 5.12.10, compilato in 64 bit
    \item ArchLinux 5.10, GCC 10.2, Qt 5.15.2, compilato in 64 bit
    \item ArchLinux 5.10, CLANG 11, Qt 5.15.2, compilato in 64 bit
\end{itemize}
Inoltre sono stati svolti dei test attraverso il software \emph{Valgrind 3.16}, riscontrando memory leak riconducibili al framework Qt.
Come IDE è stato utilizzato QtCreator 4.13.
\newline 
Per compilare su sistemi linux è necessario spostarsi nella cartella contenente i sorgenti e digitare i seguenti comandi:
\begin{lstlisting}[language=bash]
    qmake Katalog.pro
    make
\end{lstlisting}
oppure per sistemi multicore, per velocizzare la fase di compilazione anche:
\begin{lstlisting}[language=bash]
    qmake Katalog.pro
    make -j4
\end{lstlisting}
dove si può sostiturie il numero \emph{4} con il numero di core del sistema; con il parametro \emph{-j}, make si occuperà automaticamente di 
distribuire la compilazione su tutti i core disponibili. Infine per eseguire il programma basterà digitare:
\begin{lstlisting}[language=bash]
    ./Katalog
\end{lstlisting}